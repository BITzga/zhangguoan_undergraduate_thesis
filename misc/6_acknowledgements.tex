%%
% The BIThesis Template for Bachelor Graduation Thesis
%
% 北京理工大学毕业设计(论文)致谢 —— 使用 XeLaTeX 编译
%
% Copyright 2020-2021 BITNP
%
% This work may be distributed and/or modified under the
% conditions of the LaTeX Project Public License, either version 1.3
% of this license or (at your option) any later version.
% The latest version of this license is in
%   http://www.latex-project.org/lppl.txt
% and version 1.3 or later is part of all distributions of LaTeX
% version 2005/12/01 or later.
%
% This work has the LPPL maintenance status `maintained'.
%
% The Current Maintainer of this work is Feng Kaiyu.
%
% Compile with: xelatex -> biber -> xelatex -> xelatex

\unnumchapter{致~~~~谢}
\renewcommand{\thechapter}{致谢}

\ctexset{
  section/number = \arabic{section}
}

% 致谢部分尽量不使用 \subsection 二级标题,只使用 \section 一级标题

时光飞逝,如白驹过隙,转眼间我的本科求学生涯已经接近尾声。
回顾这四年的时光,许多事情在记忆里都还很深刻。
从一开始乘飞机从两千公里外的家乡来到北京求学,这时候的回忆是懵懂的、天真的。
对学业的认知还停留在绩点上,单纯地认为绩点可以决定一切。
直到第二年身边重要的人离开,我才被迫独立起来,变得更加成熟。
这时候我学会了和自己打交道,也改变了高中以来根深蒂固的想法,对学业的认知变成了更多维度的。
也就是这一年,我确定了自己的职业规划——选择找工作而不是考研。
因为是逆向而行,身边的同学都在考研,只有少数人选择找工作,我的心里其实是没有底气的。
后面的时间里,就是在想办法履行对自我的承诺,学习工作技能、找实习、实习、找工作。
还好,最后如愿以偿,我履行了一年前自己的承诺。

人生如逆旅,我亦是行人。四年的时光里,许多人曾在我身边停留,就如我在人生这座旅店停留一般。
我感谢和感恩这段时光。能遇见这些人,是我生命中的缘分与幸运。

在此,感谢我的祖母,是她陪伴了我的童年,是她教会了我是非善恶。
她的一言一行不断地影响着我,让我即使是在深陷“人性沼泽”的时候,也会坚定地选择做善良的事,选择成为善良的人。

其次,我要感谢我的初中老师林爱珠。感谢她的鼓励和帮助,让我不再迷茫。

感谢回忆里高二九班、高三九班的所有同学,感谢他们在高中时的包容与理解。
也感谢这一份长存的友谊,是它给了我许多鼓励,让我度过了大学里许多艰难的时光。

感谢在北理认识的许多朋友。感谢骑行北理、青藤田径社、风信子的朋友们。也感谢同专业的同学宋学长、龚海龙...

感谢北理的金旭亮老师和我实习时的导师张全龙,是他们教会了我许多职业技能,让我在求职路上更有底气。

感谢清华的向勇老师、华科的邵志远老师、北理的陆慧梅老师。感谢他们对我的毕业设计工作的指导与帮助。


