%%
% The BIThesis Template for Bachelor Graduation Thesis
%
% 北京理工大学毕业设计(论文)结论 —— 使用 XeLaTeX 编译
%
% Copyright 2020-2021 BITNP
%
% This work may be distributed and/or modified under the
% conditions of the LaTeX Project Public License, either version 1.3
% of this license or (at your option) any later version.
% The latest version of this license is in
%   http://www.latex-project.org/lppl.txt
% and version 1.3 or later is part of all distributions of LaTeX
% version 2005/12/01 or later.
%
% This work has the LPPL maintenance status `maintained'.
%
% The Current Maintainer of this work is Feng Kaiyu.
%
% Compile with: xelatex -> biber -> xelatex -> xelatex

\unnumchapter{结~~~~论}
\renewcommand{\thechapter}{结论}

\ctexset{
  section/number = \arabic{section}
}

% 结论部分尽量不使用 \subsection 二级标题,只使用 \section 一级标题

% 这里插入一个参考文献,仅作参考

本文从实际出发,考虑实验教学的经济成本、环境搭建的学习成本,
对现有的代理内核操作系统实验PKE进行了改造,
最总将其移植到K210开发板上,并完成了实验改进。

整个过程从原有代理内核操作系统实验的实际出发,
分析其在物理环境下进行实验的经济成本、环境搭建的成本。
逐步梳理出目前存在的痛点,并提出解决痛点的需求。
再对目前实验改造的需求进行需求分析。
需求分析结束后,本文提出了实验移植和改进的技术方案。
然后本文从环境搭建、移植的具体实现详细阐述了技术方案。
接着本文给出了代理内核操作系统实验在移植后的参考实现,
列举了九个实验的参考实现,并分别给出了代码和说明。
最后,本文给出了代理内核移植后的代码仓库的GitHub地址,
也给出了实验指导书的gitbook地址。

虽然本文完成了代理内核实验在K210上的移植和改进,
但其中还有很多可改进的点。
例如,由于时间关系,我们并未在移植后的PKE中实现文件系统。
因为没有文件系统,PKE内核在无法通过文件的形式加载用户进程。
只能通过一并编译用户程序与内核、再通过内存加载的形式加载用户进程。
这种方式是较为繁琐的,实验的内核代码或是用户程序代码的改动,
都会引起用户程序的入口点变化,
这种变化需要我们手动同步到kernel/config.h里的宏定义里。

这个痛点可以用两种方式解决,一个是直接实现文件系统,
将PKE加载用户进程的方式改为文件系统加载的方式。
另一个方式是通过编写自动化编译脚本,
使用脚本自动获取用户程序的入口点,
并将其同步到kernel/config.h里的宏定义里。
