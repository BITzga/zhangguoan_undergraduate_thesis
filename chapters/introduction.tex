% 第一章

\chapter{绪论}

\section{研究工作的背景和意义}

操作系统是负责管理计算机软硬件资源的软件,
它负责管理内存、进程管理、文件系统的管理。
经过了几十年的发展,操作系统的原理、实践已经成为计算机知识的精华。
对于计算机相关专业的学生来说,操作系统课程必不可少。
无论是学习操作系统的基本原理、设计思想,还是动手实践其内容,
对于培养计算机思维来说都是必不可少的。\cite{张其亮2010操作系统课程实验教学改革探析}

在操作系统的教学中,理论教学必不可少。
除此之外,还需要增加实验教学来培养学生的工程能力,
让学生在实践中学习操作系统原理,加深对操作系统理论知识的理解。\cite{孙述和2010操作系统实验教学研究与探索}

本文的研究工作是改造代理内核操作系统实验,
将其移植到K210开发板上,并对其实验进行改进。
代理内核是精简的、轻量的操作系统内核,
它可以给用户程序提供最基本的运行环境。
它本身不具备访问外部设备的I/O实现,
而是通过代理的方式,使用宿主机Ubuntu的I/O功能。
代理内核是RISC-V开源软件生态的一部分。
基于开源的RISC-V代理内核,
华中科技大学操作系统团队开发了代理内核操作系统实验,
并编写了其实验指导手册。

代理内核操作系统实验存在着优点与缺点。
因为代理内核只需要关心内存和CPU等核心资源,
它的代码数量极少,
这非常有利于学生学习操作系统的核心原理。
但是代理内核依赖于宿主机上的其他操作系统,这会有一些弊端。
例如,在实际的操作系统教学中,代理内核的物理环境成本高昂。
代理内核是运行在64位 RISC-V CPU上的,
其依赖的宿主机操作系统Ubuntu运行在ARM物理核上。
而一块带有FPGA(用于烧录RISC-V CPU软核)和ARM物理核的开发板,
其市场平均价格在三千元以上。
这种高昂的经济成本不利于教师在物理环境上展开大规模的实验教学。
如果教学中只有软件模拟的开发环境,这将是不利于学生学习硬件相关知识的。
除此之外,因为需要烧录RISC-V CPU软核,还需要让ARM核运行Ubuntu,
代理内核的物理环境搭建成本也比较高。

因此,本文从代理内核操作系统实验的实际出发,
分析其目前存在的问题,提出了解决方案——将代理移植到基于RISC-V的、成本低廉的K210开发板上。
最终给出了移植的技术细节与实现代码,
在移植后还对实验进行了改进,并给出了实验指导书。

\section{国内外研究现状和发展态势}

RISC-V指令集是加利福尼亚大学伯克利分校在2010年提出的指令集,
它是一种基于开放架构的指令集,也是一种精简指令集。\cite{雷思磊2017RISC}
它具有许多优点。
首先,RISC-V吸取了其他指令集的优点,避免了其他指令集的缺点。
RISC-V的指令架构是稳定的,这也保证了其指令的精简程度。
与RISC-V相反的是X86,X86指令集自诞生以来,其指令数量不断增长,
从一开始的八十条,增长到了一千多条。
最终在2015年达到了1978年的16倍。
其次,RISC-V是开放的。RISC-V被提出后,其开源生态不断完善,
代理内核(RISC-V Proxy Kernel)就是属于RISC-V生态的一部分。
这有利于后续研究者在前人的工作基础上进行研究。\cite{胡振波2019RISC}

代理内核的提出是为了支持受I/O限制的RISC-V CPU实现,
代理内核将I/O相关的系统调用处理代理到了宿主机的其他操作系统上。
在不关心I/O的条件下,
这有利于快速验证RISC-V CPU实现的功能。
代理内核项目(riscv-pk)最早是在2011年创建的,它属于RISC-V开源软件生态的一部分。

代理内核操作系统实验是华中科技大学于2021年发布的,
它是基于代理内核的思想和代理内核项目(riscv-pk)部分实现代码开发完成的。
本文就是基于此实验,对其进行了改进,并将其移植到了K210上。

近些年,将操作系统内核移植到K210的工作有很多。\cite{孙卫真2021基于}
例如在2020年,南开大学操作系统团队将清华大学的ucore操作系统移植到了K210上。
在2021年,华中科技大学操作系统团队将MIT的xv6操作系统移植到了K210上。
这些移植工作都可以给我们当作参考资料使用,但是我们不能照搬其移植过程。
因为代理内核的代理I/O机制使其区别于普通的操作系统内核,
所以我们需要逐步分析代理内核操作系统实验的源码,得出属于其移植工作的解决方案。



\section{本文的主要贡献与创新}

本文介绍了代理内核操作系统实验在K210嵌入式平台的移植过程,
最终改进了代理内核操作系统实验。本文的主要贡献与创新如下:

1.在物理环境经济成本上,本文对PKE移植前后的物理开发环境进行了分析,
评价了技术可行性,并提出了详细的技术方案并实现。
最后将代理内核操作系统实验的物理环境经济成本降至原先的三十分之一。
这有利于该操作系统实验展开大规模的物理环境教学。

2.在实验学习成本方面,
本文降低了代理内核的环境搭建成本,
让实验操作者免去实验搭建的繁琐细节。
此外,本文还对代理内核实验进行了改进,
如提供更加易用的API,增加PKE代码的可移植性,
降低了实验操作者的开发成本和学习成本。

\section{本文的结构安排}

本文的结构安排如下:

在第一章,本文主要介绍了代理内核操作系统PKE移植K210的背景和意义。
本文先是介绍了代理内核操作系统实验目前存在的问题,并给出了大致的解决方向。
然后介绍了当前操作系统相关实验的研究现状和发展态势。最后描述了本文的大致结构。

第二章,本文介绍了技术方案中涉及到的理论基础和相关技术。
并列举了其他操作系统实验移植到K210的参考方案。

第三章,本文分析了代理内核操作系统实验在移植前后的总体设计,
如系统架构、主要功能模块、执行流程等。
除此之外,本文还对比了代理内核移植前后的开发环境,得出了移植工作的预期收益。
最后给出了移植K210的技术方案。

第四章,本文给出了代理内核操作系统实验移植K210的环境搭建过程。
先是从软件环境方面描述了其过程,如编译工具链准备、
烧录K210工具准备、串口通讯工具准备。然后提出了K210的硬件环境要求。

第五章,本文展示了移植代理内核操作系统实验的技术细节。
先是向读者描述引入RustSBI的背景,再描述引入RustSBI引起的编译、启动流程改造。
然后给出了在K210上的驱动程序与接口移植的细节,
最后给出了PKE在K210上加载用户进程的技术方案。

第六章,本文给出了代理内核操作系统实验的九个基础实验的参考实现,
并在最后给出了相关的代码库链接,实验指导书链接。


