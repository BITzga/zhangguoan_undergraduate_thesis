% 第一章

\chapter{绪论}

\section{研究工作的背景和意义}

\section{国内外研究现状和发展态势}

\section{本文的主要贡献与创新}

本文介绍了代理内核操作系统实验在K210嵌入式平台的移植过程,
最终改进了代理内核操作系统实验。本文的主要贡献与创新如下:

1.在物理环境经济成本上,本文对PKE移植前后的物理开发环境进行了分析,
评价了技术可行性,并提出了详细的技术方案并实现。
最后将代理内核操作系统实验的物理环境经济成本降至原先的三十分之一。
这有利于该操作系统实验展开大规模的物理环境教学。

2.在实验学习成本方面,
本文降低了代理内核的环境搭建成本,
让实验操作者免去实验搭建的繁琐细节。
此外,本文还对代理内核实验进行了改进,
如提供更加易用的API,增加PKE代码的可移植性,
降低了实验操作者的开发成本和学习成本。

\section{本文的结构安排}

本文的结构安排如下:

在第一章,本文主要介绍了代理内核操作系统PKE移植K210的背景和意义。
本文先是介绍了代理内核操作系统实验目前存在的问题,并给出了大致的解决方向。
然后介绍了当前操作系统相关实验的研究现状和发展态势。最后描述了本文的大致结构。

第二章,本文介绍了技术方案中涉及到的理论基础和相关技术。
并列举了其他操作系统实验移植到K210的参考方案。

第三章,本文分析了代理内核操作系统实验在移植前后的总体设计,
如系统架构、主要功能模块、执行流程等。
除此之外,本文还对比了代理内核移植前后的开发环境,得出了移植工作的预期收益。
最后给出了移植K210的技术方案。

第四章,本文给出了代理内核操作系统实验移植K210的环境搭建过程。
先是从软件环境方面描述了其过程,如编译工具链准备、
烧录K210工具准备、串口通讯工具准备。然后提出了K210的硬件环境要求。

第五章,本文展示了移植代理内核操作系统实验的技术细节。
先是向读者描述引入RustSBI的背景,再描述引入RustSBI引起的编译、启动流程改造。
然后给出了在K210上的驱动程序与接口移植的细节,
最后给出了PKE在K210上加载用户进程的技术方案。

第六章,本文给出了代理内核操作系统实验的九个基础实验的参考实现,
并在最后给出了相关的代码库链接,实验指导书链接。


