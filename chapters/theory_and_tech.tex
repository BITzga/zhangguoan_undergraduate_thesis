%%
% The BIThesis Template for Bachelor Graduation Thesis
%
% 北京理工大学毕业设计(论文)第一章节 —— 使用 XeLaTeX 编译
%
% Copyright 2020-2021 BITNP
%
% This work may be distributed and/or modified under the
% conditions of the LaTeX Project Public License, either version 1.3
% of this license or (at your option) any later version.
% The latest version of this license is in
%   http://www.latex-project.org/lppl.txt
% and version 1.3 or later is part of all distributions of LaTeX
% version 2005/12/01 or later.
%
% This work has the LPPL maintenance status `maintained'.
%
% The Current Maintainer of this work is Feng Kaiyu.
%
% 第二章

\chapter{理论基础及相关技术}

\section{代理内核操作系统实验}

\subsection{代理内核的概念}

代理内核(Proxy Kernel)是一种特殊的操作系统内核。
代理内核系统是由代理内核和Host主机的操作系统Ubuntu组成的。
代理内核与Host主机的操作系统之间使用HTIF接口进行通信。

代理内核并不是一个独立的操作系统,它虽然拥有IO功能,但它不具备IO的独立实现。
它的IO功能实现依赖于Host主机的操作系统Ubuntu。
也就是说,代理内核Proxy Kernel与Host主机的操作系统Ubuntu是并行在运行的。
它们之间通过HTIF(Host Target Interface)通信。
当代理内核需要进行IO时,代理内核就通过HTIF调用Host主机的操作系统Ubuntu的IO接口,以达到IO的目的。

\subsection{代理内核的思想}

\subsection{代理内核实验的基本介绍}

\section{RISC-V新型开放指令集和精简指令集介绍}

\subsection{RISC-V的基本介绍}

\subsection{RISC-V的特权级}

\subsection{RISC-V的中断、异常委托}

\section{现有K210板子内核移植工作的参考}

\subsection{K210的基本信息}

\subsection{uCore的移植过程}

