%%
% The BIThesis Template for Bachelor Graduation Thesis
%
% 北京理工大学毕业设计(论文)中英文摘要 —— 使用 XeLaTeX 编译
%
% Copyright 2020-2021 BITNP
%
% This work may be distributed and/or modified under the
% conditions of the LaTeX Project Public License, either version 1.3
% of this license or (at your option) any later version.
% The latest version of this license is in
%   http://www.latex-project.org/lppl.txt
% and version 1.3 or later is part of all distributions of LaTeX
% version 2005/12/01 or later.
%
% This work has the LPPL maintenance status `maintained'.
%
% The Current Maintainer of this work is Feng Kaiyu.

% 中英文摘要章节
\zihao{-4}
\vspace*{-11mm}

\begin{center}
  \heiti\zihao{-2}\textbf{\thesisTitle}
\end{center}

\vspace*{2mm}

{\let\clearpage\relax \chapter*{\textmd{摘~~~~要}}}
\addcontentsline{toc}{chapter}{摘~~~~要}
\setcounter{page}{1}

\vspace*{1mm}

\setstretch{1.53}
\setlength{\parskip}{0em}

% 中文摘要正文从这里开始
操作系统课程是计算机专业的重要专业基础课,
其对于培养学生的工程能力与设计能力有着很大的意义。
而操作系统实验部分更是课程的核心内容,
该部分可以让学生在实践中学习操作系统的算法、工程实现、设计思想。

代理内核是不具备独立I/O实现的操作系统内核,
它的提出了为了支持受到I/O限制的RISC-V CPU。
它的所依赖的I/O功能都代理给了宿主机上的操作系统。
它具有轻量、代码精简的优点。

本文是基于华中科技大学的代理内核操作系统实验的移植和改进。
本文从源代码层面,经济成本与学习成本方面,
分析了代理内核实验在教学过程中存在的优点与缺点。
对于其目前存在的经济成本高、环境搭建复杂等缺点,本文提出了实验改进的需求。
然后进行需求分析,并给出了移植到K210开发板上的解决方案。
最后本文给出了移植代理内核操作系统实验到K210开发板上的具体实现,
并编写了相关的实验指导书。完成了对代理内核操作系统实验的移植与改进。

\vspace{4ex}\noindent\textbf{\heiti 关键词:操作系统;内核;代理内核;实验改进;K210;移植}
\newpage

% 英文摘要章节
\vspace*{-2mm}

\begin{spacing}{0.95}
  \centering
  \heiti\zihao{3}\textbf{\thesisTitleEN}
\end{spacing}

\vspace*{17mm}

{\let\clearpage\relax \chapter*{
  \zihao{-3}\textmd{Abstract}\vskip -3bp}}
\addcontentsline{toc}{chapter}{Abstract}
\setcounter{page}{2}

\setstretch{1.53}
\setlength{\parskip}{0em}

% 英文摘要正文从这里开始
The operating system course is an important professional course for the students major in Computer Science,
It is of great significance to develop students' engineering ability and design ability.
The operating system experiment is the core content of the course,
This part allows students to learn the algorithm, 
engineering implementation and design idea of the operating system in practice.

The RISC-V Proxy Kernel is an operating system without independent I/O implementation.
It is designed to support RISC-V CPU without I/O capability.
It delegates I/O to a host computer for handling I/O calls.
It has the advantages of light weight and simplified code.

This paper is based on the transplantation and improvement of the Proxy Kernel operating system experiment of Huazhong University of Science and Technology.
From the perspective of source code, economic cost and learning cost,
we analyze the advantages and disadvantages of Proxy Kernel experiment in the teaching process.
This paper puts forward the disadvantages of high cost and complex experimental environment.
Then the requirements analysis is carried out, and the solution which transplanted PKE to K210 development board is given.
Finally, this paper gives the concrete implementation of the solution,
and then gives the tutorial of the experiment.

\vspace{3ex}\noindent\textbf{Key Words: Operating System;Proxy Kernel;Improvement;K210;Transplantation}
\newpage
