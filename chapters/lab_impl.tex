\chapter{代理内核实验的参考实现}

在本章节,我们将给出移植K210后的代理内核实验的参考实现。
每个实验实现的阐述结构大致相同,
先从实验预期开始,描述实验的目标和预期效果。
最后我们会给出参考实现的代码和实现过程描述。

\section{系统调用}

\subsection{实验预期}

给定用户态应用user/app\_helloworld.c。

\begin{lstlisting}[caption={用户态应用app\_helloworld.c}, label={lst:app_helloworld}, language=C]
    #include "user_lib.h"

    int main(void) {
        printu("Hello world!\n");
        exit(0);
    }   
\end{lstlisting}

实验的目标是PKE能够加载并成功运行它。
实验预期是应用通过系统调用,成功打印出“Hello world!”,并正常退出。

\subsection{具体实现}

如上的用户态代码很简单,只有两行。第一行是用户态的打印函数printu()。
第二行是退出进程的函数exit()。
关于printu()和exit()函数的定义在user\_lib.c。
接下来我们看看这两个函数的具体实现:

\begin{lstlisting}[caption={printu与exit的实现}, label={lst:printu_exit}, language=C]
    int printu(const char *s, ...) {
        va_list vl;
        va_start(vl, s);
        char out[256];  // fixed buffer size.
        int res = vsnprintf(out, sizeof(out), s, vl);
        va_end(vl);
        const char *buf = out;
        size_t n = res < sizeof(out) ? res : sizeof(out);

        return do_user_call(SYS_user_print, (uint64) buf, n, 0, 0, 0, 0, 0);
    }

    int exit(int code) {
        return do_user_call(SYS_user_exit, code, 0, 0, 0, 0, 0, 0);
    }   
\end{lstlisting}

观察发现,printu()和exit()函数的实现都是简单处理了输入的参数,
然后就调用了do\_user\_call()函数。他们俩都是通过系统调用来获取自己所需的功能。
那么结论已经很明显,
我们只需要在PKE的内核代码中实现printu()和exit()的系统调用即可。

观察内核代码发现,printu()和exit()的系统调用都已经实现,
系统调用部分只有handle\_syscall()并未实现。该函数的功能是让用户进程的pc值增加4。
之所以增加4,是因为RISC-V的指令是32位的,32位代表着4个字节。
然后调用系统调用,最终将返回值保存在a0寄存器中。接下来我们给出它的实现代码。

\begin{lstlisting}[caption={handle\_syscall的实现}, label={lst:handle_syscall}, language=C]
static void handle_syscall(trapframe *tf) {
    tf->epc += 4;
    tf->regs.a0 = do_syscall(tf->regs.a0, tf->regs.a1,
                                tf->regs.a2, tf->regs.a3, 
                                tf->regs.a4, tf->regs.a5, 
                                tf->regs.a6, tf->regs.a7);
}
\end{lstlisting}

将实验要求实现后,我们可以在K210上运行实验代码,得到如下结果。

\begin{lstlisting}[caption={lab1实验结果}, label={lst:lab1_result}]
    [rustsbi] Version 0.1.0
    .______       __    __      _______.___________.  _______..______   __
    |   _  \     |  |  |  |    /       |           | /       ||   _  \ |  |
    |  |_)  |    |  |  |  |   |   (----`---|  |----`|   (----`|  |_)  ||  |
    |      /     |  |  |  |    \   \       |  |      \   \    |   _  < |  |
    |  |\  \----.|  `--'  |.----)   |      |  |  .----)   |   |  |_)  ||  |
    | _| `._____| \______/ |_______/       |__|  |_______/    |______/ |__|
    
    [rustsbi] Platform: K210
    [rustsbi] misa: RV64ACDFIMSU
    [rustsbi] mideleg: 0x222
    [rustsbi] medeleg: 0x1ab
    [rustsbi] Kernel entry: 0x80020000
    Enter supervisor mode...
    
    Application program entry point (virtual address): 0x00000000800206b2
    
    Switch to user mode...
    
    Hello world!
    
    User exit with code:0.
    
    System is shutting down with exit code 0.
    
    [rustsbi] todo: shutdown all harts on k210; program halt
        
\end{lstlisting}

\section{异常处理}

\subsection{实验预期}

给定用户态应用user/app\_illegal\_instruction.c

\begin{lstlisting}[caption={用户态应用app\_illegal\_instruction.c}, label={lst:app_illegal_instruction}, language=C]
    int main(void) {
        printu("Going to hack the system by running privilege instructions.\n");
        asm volatile("csrw sscratch, 0");
        exit(0);
    }
\end{lstlisting}

我们可以看到,用户态程序执行了非法指令csrw。
这个指令的执行在U模式是不被允许的,
在U模式执行这个指令会产生非法指令异常。
我们的实验预期是PKE能够捕获执行这个指令时产生的非法指令异常,
并进行简单处理。

\subsection{实验分析}

由于RustSBI-K210支持包对K210进行了兼容性的处理。
K210实现的RISC-V指令集版本是1.9.1,而最新的RISC-V指令集版本是1.11.0。
为了能让K210能够运行高版本的RISC-V指令,
该支持包捕获了高版本指令产生的异常,然后将其转换为低版本的指令。
这种机制使得我们可以在K210上运行最新版的RISC-V指令。

为了保持这种兼容性,指令异常只能由M模式的RustSBI处理,
我们不能将指令异常托管给S模式的PKE。
所以此次实验,我们没有办法在PKE中完成,
只能交给RustSBI处理。
所幸,RustSBI已经实现了非法指令的捕获与处理,
当用户程序执行非法指令时,RustSBI会捕获异常,
并且产生panic,输出panic信息。这和原来实验的预期是一致的。

此次实验不需要我们改动PKE代码,就可以得到预期效果:

\begin{lstlisting}[caption={lab2实验结果}, label={lst:app_illegal_instruction_result}]
    [rustsbi] Version 0.1.0
    .______       __    __      _______.___________.  _______..______   __
    |   _  \     |  |  |  |    /       |           | /       ||   _  \ |  |
    |  |_)  |    |  |  |  |   |   (----`---|  |----`|   (----`|  |_)  ||  |
    |      /     |  |  |  |    \   \       |  |      \   \    |   _  < |  |
    |  |\  \----.|  `--'  |.----)   |      |  |  .----)   |   |  |_)  ||  |
    | _| `._____| \______/ |_______/       |__|  |_______/    |______/ |__|
    
    [rustsbi] Platform: K210
    [rustsbi] misa: RV64ACDFIMSU
    [rustsbi] mideleg: 0x222
    [rustsbi] medeleg: 0x1ab
    [rustsbi] Kernel entry: 0x80020000
    Enter supervisor mode...
    
    Application program entry point (virtual address): 0x0000000080020788
    
    Switch to user mode...
    
    Going to hack the system by running privilege instructions.
    
    [rustsbi] panicked at 'invalid instruction, mepc: 0000000080020798, instruction: 14005073', platform/k210/src/main.rs:522:17
        
\end{lstlisting}

\section{定时器中断}

\subsection{实验预期}

给定用户态应用user/app\_long\_loop.c

\begin{lstlisting}[caption={用户态应用app\_long\_loop.c}, label={lst:app_long_loop}, language=C]
    #include "user_lib.h"

    int main(void) {
      printu("Hello world!\n");
      int i;
      for (i = 0; i < 100000000; ++i) {
        if (i % 5000000 == 0) printu("wait %d\n", i);
      }
      exit(0);
    }
\end{lstlisting}

实验操作者需要初始化时钟中断,
设置定时器中断的时间间隔,
并且编写时钟中断处理程序。
最终让用户程序执行循环的同时,
定时地打印出定时器中断自增的次数。

\subsection{具体实现}

\begin{lstlisting}[caption={定时器中断处理程序}, label={lst:app_timer_handler}, language=C]

    //时钟初始化部分已在kernel.c中的s_start()中完成。
    void handle_stimer_trap() {
        sprint("Ticks %d\n", ++g_ticks);
        clock_set_next_event();
    }    
\end{lstlisting}

我们的实现逻辑很简单,就是让定时器中断的次数自增,并打印输出。
然后设置下一次定时器中断的时间。

\section{虚拟地址和物理地址的转换}

\subsection{实验预期}

给定用户态应用user/app\_helloworld\_no\_lds.c

\begin{lstlisting}[caption={用户态应用app\_helloworld\_no\_lds.c}, label={lst:app_helloworld_no_lds}, language=C]
    #include "user_lib.h"
    
    int main(void) {
      printu("Hello world!\n");
      exit(0);
    }    
\end{lstlisting}

此时的用户程序是没有链接脚本文件的。
用户程序在编译和链接时并未指定好程序中符号的地址。
与此同时,PKE已经开启虚拟内存,
用户进程的物理内存空间也被映射到虚拟内存上了。

用户进程在调用printu时,它会通过ecall指令,切换到拥有更高特权级的S模式的内核,
让内核完成打印输出。需要指出的是,用户进程传入内核的地址参数是虚拟地址,
内核需要将该虚拟地址转换成物理地址才能访问字符串“Hello world!”。

所以我们的实验预期是内核成功转换用户进程传入的虚拟地址,并完成打印输出。

\subsection{具体实现}

\begin{lstlisting}[caption={虚拟地址到物理地址的转换}, label={lst:app_virt_to_phys}, language=C]
    ssize_t sys_user_print(const char* buf, size_t n) {
        //buf is an address in user space on user stack,
        //so we have to transfer it into phisical address (kernel is running in direct mapping).
        assert( current );
        char* pa = (char*)user_va_to_pa((pagetable_t)(current->pagetable), (void*)buf);
        sprint(pa);
        return 0;
    }    

    void *user_va_to_pa(pagetable_t page_dir, void *va) {
        uint64 va0 = (uint64) va;
        pte_t *PTE = page_walk(page_dir, va0, 0);
        uint64 pa = 0;
        if (PTE) {
            pa = PTE2PA(*PTE) + (va0 & ((1 << PGSHIFT) - 1));
            return (void *) pa;
        } else {
            return NULL;
        }
    }    
\end{lstlisting}

虚拟地址到物理地址的转换逻辑很简单,
首先是通过调用page\_walk()函数,查找虚拟地址的页表项。
找到页表项以后,再通过PTE2PA()函数和一些移位操作获得物理地址。

\section{基本的内存管理}

\subsection{实验预期}

给定用户态应用user/app\_naive\_malloc.c

\begin{lstlisting}[caption={用户态应用app\_naive\_malloc.c}, label={lst:app_naive_malloc}, language=C]
    #include "user_lib.h"
    
    struct my_structure {
      char c;
      int n;
    };
    
    int main(void) {
      struct my_structure* s = (struct my_structure*)naive_malloc();
      s->c = 'a';
      s->n = 1;

      printu("s: %lx, {%c %d}\n", s, s->c, s->n);
    
      naive_free(s);
      exit(0);
    }   
\end{lstlisting}

如用户程序所示,我们的PKE需要支持内存的分配和回收。
而内存的分配逻辑在PKE中已经实现好了,我们现在还需要实现内存的回收。
也就是实现naive\_free()函数。


\subsection{具体实现}

\begin{lstlisting}
    uint64 sys_user_free_page(uint64 va) {
        user_vm_unmap((pagetable_t)current->pagetable, va, PGSIZE, 1);
        return 0;
    }  
    
    void user_vm_unmap(pagetable_t page_dir, uint64 va, uint64 size, int free) {
        if (free) {
            pte_t *PTE = page_walk(page_dir, va, 0);
            if (PTE != NULL) {
                free_page(user_va_to_pa(page_dir, (void *) va));
                *PTE = *PTE & (~PTE_V);
            } else {
                sprint("PTE of free page is NULL\n");
            }
        }
    }    
\end{lstlisting}

内存回收的逻辑稍微复杂一些,
我们首先是根据根页表项和虚拟地址去调用page\_walk()函数,
得到需要回收的虚拟地址对应的页表项。
然后我们将页表项的PTE\_V位置0,这意味着此页表项将变成无效页表项。
与此同时,我们通过free\_page()函数将物理页面回收。
在这里我们只考虑一个物理页。

\section{栈空间不足与缺页异常}

\subsection{实验预期}

给定应用user/app\_sum\_sequence.c

\begin{lstlisting}[caption={用户态应用app\_sum\_sequence.c}, label={lst:app_sum_sequence}, language=C]
    #include "user_lib.h"
    #include "../util/types.h"

    uint64 sum_sequence(uint64 n) {
      if (n == 0)
        return 0;
      else
        return sum_sequence( n-1 ) + n;
    }
    
    int main(void) {
      uint64 n = 1000;
      printu("Summation of an arithmetic sequence from 0 to %ld is: %ld \n", n, sum_sequence(n) );
      exit(0);
    }
        
\end{lstlisting}

如上代码所示,
用户程序执行了一个递归求等差数列的函数,递归的层数是1000层。
显然,这么深的递归层数,很容易导致栈溢出,从而导致缺页异常。

所以我们的实验预期是PKE能够捕获缺页异常,
并且能够为用户进程的栈空间进行一定的扩容。
让该用户进程成功运行。

\subsection{具体实现}

\begin{lstlisting}[caption={handle\_page\_fault}, label={lst:handle_page_fault}, language=C]
    void handle_user_page_fault(uint64 mcause, uint64 sepc, uint64 stval) {
        sprint("handle_page_fault: %lx\n", stval);
        switch (mcause) {
            case CAUSE_STORE_PAGE_FAULT:
                if (stval < USER_STACK_TOP 
                        && (stval - USER_STACK_TOP) >= 20 * (STACK_SIZE)) {
                    user_vm_map((pagetable_t) current->pagetable,
                                    stval - stval % PGSIZE, 
                                    PGSIZE, 
                                    (uint64) alloc_page(),
                                    prot_to_type(PROT_WRITE | PROT_READ, 1));
                } else {
                    panic("handle_page_fault: stack overflow\n");
                }
                break;
            default:
                sprint("unknown page fault.\n");
                break;
        }
    }    
\end{lstlisting}

\section{创建子进程fork的实现}

\subsection{实验预期}

给定应用user/app\_naive\_fork.c

\begin{lstlisting}[caption={用户态应用app\_naive\_fork.c}, label={lst:app_naive_fork}, language=C]
    #include "../user/user_lib.h"
    #include "../util/types.h"
    
    int main(void) {
        uint64 pid = fork();
        if (pid == 0) {
            printu("Child: Hello world!\n");
        } else {
            printu("Parent: Hello world! child id %ld\n", pid);
        }
    
        exit(0);
    }
       
\end{lstlisting}

如用户代码所示,PKE需要创建一个用户进程的子进程,然后调度运行两者。

\subsection{具体实现}

如下所示,我们在kernel/process.c中补全了do\_fork()函数。

\begin{lstlisting}[caption={do\_fork}, label={lst:do_fork}, language=C]
    //                panic("You need to implement the code segment mapping of child in lab3_1.\n");
    for (int j = 0; j < parent->mapped_info[i].npages; j++) {
        map_pages(child->pagetable, 
                    parent->mapped_info[i].va + j * PGSIZE, 
                    (uint64) _etext - USER_PROGRAM_ENTRY,
                    USER_PROGRAM_ENTRY, 
                    prot_to_type(PROT_WRITE | PROT_READ | PROT_EXEC, 1));

        sprint("do_fork map code segment at pa:%lx of parent to child at va:%lx.\n",
               USER_PROGRAM_ENTRY, parent->mapped_info[i].va + j * PGSIZE);
    }    
\end{lstlisting}

\section{进程的控制权交接}

\subsection{实验预期}

给定应用user/app\_yield.c

\begin{lstlisting}[caption={用户态应用app\_yield.c}, label={lst:app_yield}, language=C]
    #include "../user/user_lib.h"
    #include "../util/types.h"
    
    int main(void) {
        uint64 pid = fork();
        uint64 rounds = 0xffff;
        if (pid == 0) {
            printu("Child: Hello world! \n");
            for (uint64 i = 0; i < rounds; ++i) {
                if (i % 10000 == 0) {
                    printu("Child running %ld \n", i);
                    yield();
                }
            }
        } else {
            printu("Parent: Hello world! \n");
            for (uint64 i = 0; i < rounds; ++i) {
                if (i % 10000 == 0) {
                    printu("Parent running %ld \n", i);
                    yield();
                }
            }
        }
    
        exit(0);
        return 0;
    }
    
\end{lstlisting}

如用户代码所示,父子进程都轮流执行了yield()函数。
yield()函数是一个可以交出进程自己的CPU执行权的函数。
yield()函数是由PKE内核实现的系统调用功能之一。

我们的实验预期是,父子进程之间的调度是可以交替的,
也就是说父子进程之间需要轮流打印出他们对应的字符串。

\subsection{具体实现}

如下,本文给出了yield()函数的实现。实现逻辑较为简单,就不再具体阐述了。

\begin{lstlisting}[caption={yield}, label={lst:yield}, language=C]
    ssize_t sys_user_yield() {
        assert(current);
        current->status = READY;
        insert_to_ready_queue(current);
        schedule();
    
        return 0;
    }    
\end{lstlisting}

\section{进程的时间片调度}

\subsection{实验预期}

给定应用user/app\_two\_long\_loops.c

\begin{lstlisting}[caption={用户态应用app\_two\_long\_loops.c}, label={lst:app_two_long_loops}, language=C]
   #include "../user/user_lib.h"
   #include "../util/types.h"
   
   int main(void) {
       uint64 pid = fork();
       uint64 rounds = 100000000;
       uint64 interval = 10000000;
       uint64 a = 0;
       if (pid == 0) {
           printu("Child: Hello world! \n");
           for (uint64 i = 0; i < rounds; ++i) {
               if (i % interval == 0) printu("Child running %ld \n", i);
           }
       } else {
           printu("Parent: Hello world! \n");
           for (uint64 i = 0; i < rounds; ++i) {
               if (i % interval == 0) printu("Parent running %ld \n", i);
           }
       }
   
       exit(0);
       return 0;
   }
       
\end{lstlisting}

如上代码所示,
我们的用户父子进程不再主动调用yield()函数交出自己的CPU执行权。
而是由PKE内核利用时间片轮询的方式来调度父子进程,
定时地切换父子进程的执行权。

所以我们的实验预期是,完成PKE内核中时间片调度进程的实现,
并让父子进程交替执行。

\subsection{具体实现}

\begin{lstlisting}[caption={时间片调度}, label={lst:time_slice}, language=C]
    void smode_trap_handler(void) {
        assert(current);
        current->trapframe->epc = read_csr(sepc);
        uint64 cause = read_csr(scause);
    
        switch (cause) {
            case CAUSE_STIMER_S_TRAP:
                handle_stimer_trap();
                rrsched();
                break;
            default:
                //......省略展示此部分代码......
        }
        switch_to(current);
    }

    void rrsched() {
        assert(current);
        if (current->tick_count < TIME_SLICE_LEN) {
            current->tick_count++;
        } else {
            current->tick_count = 0;
            current->status = READY;
            insert_to_ready_queue(current);
            schedule();
        }
    }    
\end{lstlisting}

时间片调度进程的实现,是通过设置定时器中断,
当S模式的PKE拦截到定时器中断时,
就会执行中断处理程序,处理程序其中就包括了进程的切换。

\section{实验资料的管理}

在我们移植实验的过程中,我们使用了git来管理PKE移植的代码,
并且将移植后的代码放置在了GitHub仓库中。GitHub的仓库链接如下:

\href{https://github.com/BITzga/riscv64-pke-k210}{riscv64-pke-k210}。

此代码库包含18个主要分支,其中有9个分支是PKE移植前的代码分支,
这9个分支的代码保持不变,没有受到修改,是移植前的基础分支,
也可以供我们进行移植前后的代码比较。

此外还有另外9个k210前缀的分支。他们是移植后的代码分支,均已通过功能测试
,都可以被烧录和运行在K210上, 并且可以达到实验预期效果。

\begin{lstlisting}[caption={分支列表}, label={lst:branch_list_1}]
    remotes/origin/k210/lab1_1_syscall
    remotes/origin/k210/lab1_2_exception
    remotes/origin/k210/lab1_3_irq
    remotes/origin/k210/lab2_1_pagetable
    remotes/origin/k210/lab2_2_allocatepage
    remotes/origin/k210/lab2_3_pagefault
    remotes/origin/k210/lab3_1_fork
    remotes/origin/k210/lab3_2_yield
    remotes/origin/k210/lab3_3_rrsched
    remotes/origin/lab1_1_syscall
    remotes/origin/lab1_2_exception
    remotes/origin/lab1_3_irq
    remotes/origin/lab2_1_pagetable
    remotes/origin/lab2_2_allocatepage
    remotes/origin/lab2_3_pagefault
    remotes/origin/lab3_1_fork
    remotes/origin/lab3_2_yield
    remotes/origin/lab3_3_rrsched
\end{lstlisting}

此外,我们在移植开发工作完成以后,
还编写了相关的实验指导书,并放置在了gitbook上。
由于时间关系,我们的实验指导书的并没有达到很高的完成程度,
后续我们将对其进行完善。
这里我们先给出实验指导书的gitbook链接:

\href{https://1621184422.gitbook.io/pke-k210-tutorial}{riscv-pke-k210-tutorial}