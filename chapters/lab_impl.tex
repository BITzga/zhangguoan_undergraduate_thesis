\chapter{代理内核实验的参考实现}

在本章节,我们将给出移植K210后的代理内核实验的参考实现。
每个实验实现的阐述结构大致相同,
先从实验预期开始,描述实验的目标和预期效果。
然后我们会给出参考实现的代码和实现过程描述。
最后给出实验执行流程的总结。

\section{系统调用}

给定用户态应用user/app\_helloworld.c。

\begin{lstlisting}[caption={用户态应用app\_helloworld.c}, label={lst:app_helloworld}, language=C]
    #include "user_lib.h"

    int main(void) {
        printu("Hello world!\n");
        exit(0);
    }   
\end{lstlisting}

\subsection{实验预期}

实验的目标是PKE能够加载并成功运行它。
实验预期是应用可以打印出“Hello world!”,并正常退出。

\subsection{具体实现}

如上的用户态代码很简单,只有两行。第一行是用户态的打印函数printu()。
第二行是退出进程的函数exit()。
关于printu()和exit()函数的定义在user\_lib.c。
接下来我们看看这两个函数的具体实现:

\begin{lstlisting}[caption={printu与exit的实现}, label={lst:printu_exit}, language=C]
    int printu(const char *s, ...) {
        va_list vl;
        va_start(vl, s);
        char out[256];  // fixed buffer size.
        int res = vsnprintf(out, sizeof(out), s, vl);
        va_end(vl);
        const char *buf = out;
        size_t n = res < sizeof(out) ? res : sizeof(out);

        return do_user_call(SYS_user_print, (uint64) buf, n, 0, 0, 0, 0, 0);
    }

    int exit(int code) {
        return do_user_call(SYS_user_exit, code, 0, 0, 0, 0, 0, 0);
    }   
\end{lstlisting}

观察发现,printu()和exit()函数的实现都是简单处理了输入的参数,
然后就调用了do\_user\_call()函数。他们俩都是通过系统调用来获取自己所需的功能。
那么结论已经很明显,
我们只需要在PKE的内核代码中实现printu()和exit()的系统调用即可。

观察内核代码发现,printu()和exit()的系统调用都已经实现,
系统调用部分只有handle\_syscall()并未实现。该函数的功能是让用户进程的pc值增加4。
之所以增加4,是因为RISC-V的指令是32位的,32位代表着4个字节。
然后调用系统调用,最终将返回值保存在a0寄存器中。接下来我们给出它的实现代码。

\begin{lstlisting}[caption={handle\_syscall的实现}, label={lst:handle_syscall}, language=C]
static void handle_syscall(trapframe *tf) {
    tf->epc += 4;
    tf->regs.a0 = do_syscall(tf->regs.a0, tf->regs.a1,
                                tf->regs.a2, tf->regs.a3, 
                                tf->regs.a4, tf->regs.a5, 
                                tf->regs.a6, tf->regs.a7);
}
\end{lstlisting}

将实验要求实现后,我们可以在K210上运行实验代码,得到如下结果。

\begin{lstlisting}[caption={lab1实验结果}, label={lst:lab1_result}]
    [rustsbi] Version 0.1.0
    .______       __    __      _______.___________.  _______..______   __
    |   _  \     |  |  |  |    /       |           | /       ||   _  \ |  |
    |  |_)  |    |  |  |  |   |   (----`---|  |----`|   (----`|  |_)  ||  |
    |      /     |  |  |  |    \   \       |  |      \   \    |   _  < |  |
    |  |\  \----.|  `--'  |.----)   |      |  |  .----)   |   |  |_)  ||  |
    | _| `._____| \______/ |_______/       |__|  |_______/    |______/ |__|
    
    [rustsbi] Platform: K210
    [rustsbi] misa: RV64ACDFIMSU
    [rustsbi] mideleg: 0x222
    [rustsbi] medeleg: 0x1ab
    [rustsbi] Kernel entry: 0x80020000
    Enter supervisor mode...
    
    Application program entry point (virtual address): 0x00000000800206b2
    
    Switch to user mode...
    
    Hello world!
    
    User exit with code:0.
    
    System is shutting down with exit code 0.
    
    [rustsbi] todo: shutdown all harts on k210; program halt
        
\end{lstlisting}

\section{异常处理}

给定用户态应用user/app\_illegal\_instruction.c

\begin{lstlisting}[caption={用户态应用app\_illegal\_instruction.c}, label={lst:app_illegal_instruction}, language=C]
    int main(void) {
        printu("Going to hack the system by running privilege instructions.\n");
        asm volatile("csrw sscratch, 0");
        exit(0);
    }
\end{lstlisting}

\subsection{实验预期}

用户态程序执行了非法指令csrw。
PKE应当能够捕获非法指令异常,并进行简单处理。

\subsection{实验分析}

由于RustSBI-K210支持包对K210进行了兼容性的处理。
K210实现的RISC-V指令集版本是1.9.1,而最新的RISC-V指令集版本是1.11.0。
为了能让K210能够运行高版本的RISC-V指令,
该支持包捕获了高版本指令产生的异常,然后将其转换为低版本的指令。
这种机制使得我们可以在K210上运行最新版的RISC-V指令。

为了保持这种兼容性,指令异常只能由M模式的RustSBI处理,
我们不能将指令异常托管给S模式的PKE。
所以此次实验,我们没有办法在PKE中完成,
只能交给RustSBI处理。
所幸,RustSBI已经实现了非法指令的捕获与处理,
当用户程序执行非法指令时,RustSBI会捕获异常,
并且产生panic,输出panic信息。这和原来实验的预期是一致的。

此次实验不需要我们改动PKE代码,就可以得到预期效果:

\begin{lstlisting}[caption={lab2实验结果}, label={lst:app_illegal_instruction_result}]
    [rustsbi] Version 0.1.0
    .______       __    __      _______.___________.  _______..______   __
    |   _  \     |  |  |  |    /       |           | /       ||   _  \ |  |
    |  |_)  |    |  |  |  |   |   (----`---|  |----`|   (----`|  |_)  ||  |
    |      /     |  |  |  |    \   \       |  |      \   \    |   _  < |  |
    |  |\  \----.|  `--'  |.----)   |      |  |  .----)   |   |  |_)  ||  |
    | _| `._____| \______/ |_______/       |__|  |_______/    |______/ |__|
    
    [rustsbi] Platform: K210
    [rustsbi] misa: RV64ACDFIMSU
    [rustsbi] mideleg: 0x222
    [rustsbi] medeleg: 0x1ab
    [rustsbi] Kernel entry: 0x80020000
    Enter supervisor mode...
    
    Application program entry point (virtual address): 0x0000000080020788
    
    Switch to user mode...
    
    Going to hack the system by running privilege instructions.
    
    [rustsbi] panicked at 'invalid instruction, mepc: 0000000080020798, instruction: 14005073', platform/k210/src/main.rs:522:17
        
\end{lstlisting}

\section{定时器中断}
给定用户态应用user/app\_long\_loop.c

\begin{lstlisting}[caption={用户态应用app\_long\_loop.c}, label={lst:app_long_loop}, language=C]
    #include "user_lib.h"

    int main(void) {
      printu("Hello world!\n");
      int i;
      for (i = 0; i < 100000000; ++i) {
        if (i % 5000000 == 0) printu("wait %d\n", i);
      }
      exit(0);
    }
\end{lstlisting}

\subsection{实验预期}

实验操作者需要初始化时钟中断,
设置定时器中断的时间间隔,
并且编写时钟中断处理程序。
最终让用户程序执行循环的同时,
打印输出定时器中断自增的次数。

\subsection{具体实现}

\begin{lstlisting}[caption={定时器中断处理程序}, label={lst:app_timer_handler}, language=C]

    //时钟初始化部分已在kernel.c中的s_start()中完成。
    void handle_stimer_trap() {
        sprint("Ticks %d\n", ++g_ticks);
        clock_set_next_event();
    }    
\end{lstlisting}

我们的实现逻辑很简单,就是让定时器中断的次数自增,并打印输出。
然后设置下一次定时器中断的时间。

\section{虚拟地址和物理地址的转换}

给定用户态应用user/app\_helloworld\_no\_lds.c

\begin{lstlisting}[caption={用户态应用app\_helloworld\_no\_lds.c}, label={lst:app_helloworld_no_lds}, language=C]
    #include "user_lib.h"
    
    int main(void) {
      printu("Hello world!\n");
      exit(0);
    }    
\end{lstlisting}

\subsection{实验预期}

\subsection{具体实现}

\subsection{执行流程}

\section{基本的内存管理}

给定用户态应用user/app\_naive\_malloc.c

\begin{lstlisting}[caption={用户态应用app\_naive\_malloc.c}, label={lst:app_naive_malloc}, language=C]
    #include "user_lib.h"
    
    struct my_structure {
      char c;
      int n;
    };
    
    int main(void) {
      struct my_structure* s = (struct my_structure*)naive_malloc();
      s->c = 'a';
      s->n = 1;

      printu("s: %lx, {%c %d}\n", s, s->c, s->n);
    
      naive_free(s);
      exit(0);
    }   
\end{lstlisting}

\subsection{实验预期}

\subsection{具体实现}

\subsection{执行流程}

\section{栈空间不足与缺页异常}

给定应用user/app\_sum\_sequence.c

\begin{lstlisting}[caption={用户态应用app\_sum\_sequence.c}, label={lst:app_sum_sequence}, language=C]
    #include "user_lib.h"
    #include "../util/types.h"

    uint64 sum_sequence(uint64 n) {
      if (n == 0)
        return 0;
      else
        return sum_sequence( n-1 ) + n;
    }
    
    int main(void) {
      uint64 n = 1000;
      printu("Summation of an arithmetic sequence from 0 to %ld is: %ld \n", n, sum_sequence(n) );
      exit(0);
    }
        
\end{lstlisting}

\subsection{实验预期}

\subsection{具体实现}

\subsection{执行流程}

\section{创建子进程fork的实现}

给定应用user/app\_naive\_fork.c

\begin{lstlisting}[caption={用户态应用app\_naive\_fork.c}, label={lst:app_naive_fork}, language=C]
    #include "../user/user_lib.h"
    #include "../util/types.h"
    
    int main(void) {
        uint64 pid = fork();
        if (pid == 0) {
            printu("Child: Hello world!\n");
        } else {
            printu("Parent: Hello world! child id %ld\n", pid);
        }
    
        exit(0);
    }
       
\end{lstlisting}

\subsection{实验预期}

\subsection{兼容K210与改进}

\subsection{具体实现}

\subsection{执行流程}

\section{进程的控制权交接}

给定应用user/app\_yield.c

\begin{lstlisting}[caption={用户态应用app\_yield.c}, label={lst:app_yield}, language=C]
    #include "../user/user_lib.h"
    #include "../util/types.h"
    
    int main(void) {
        uint64 pid = fork();
        uint64 rounds = 0xffff;
        if (pid == 0) {
            printu("Child: Hello world! \n");
            for (uint64 i = 0; i < rounds; ++i) {
                if (i % 10000 == 0) {
                    printu("Child running %ld \n", i);
                    yield();
                }
            }
        } else {
            printu("Parent: Hello world! \n");
            for (uint64 i = 0; i < rounds; ++i) {
                if (i % 10000 == 0) {
                    printu("Parent running %ld \n", i);
                    yield();
                }
            }
        }
    
        exit(0);
        return 0;
    }
    
\end{lstlisting}

\subsection{实验预期}

\subsection{具体实现}

\subsection{执行流程}

\section{进程的时间片调度}

给定应用user/app\_two\_long\_loops.c

\begin{lstlisting}[caption={用户态应用app\_two\_long\_loops.c}, label={lst:app_two_long_loops}, language=C]
   #include "../user/user_lib.h"
   #include "../util/types.h"
   
   int main(void) {
       uint64 pid = fork();
       uint64 rounds = 100000000;
       uint64 interval = 10000000;
       uint64 a = 0;
       if (pid == 0) {
           printu("Child: Hello world! \n");
           for (uint64 i = 0; i < rounds; ++i) {
               if (i % interval == 0) printu("Child running %ld \n", i);
           }
       } else {
           printu("Parent: Hello world! \n");
           for (uint64 i = 0; i < rounds; ++i) {
               if (i % interval == 0) printu("Parent running %ld \n", i);
           }
       }
   
       exit(0);
       return 0;
   }
       
\end{lstlisting}

\subsection{实验预期}

\subsection{具体实现}

\subsection{执行流程}

\section{实验资料的编写及管理}

\subsection{实验指导书的编写}

\subsection{实验指导书的管理}

\subsection{对应代码库的管理}
你好
